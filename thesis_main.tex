%-----------------------------------
% Define document and include general packages
%-----------------------------------
% Tabellen- und Abkürzungsverzeichnis stehen normalerweise nicht im
% Inhaltsverzeichnis. Gleiches gilt für das Abkürzungsverzeichnis (siehe unten).
% Manche Dozenten bemängeln das. Die Optionen 'listof=totoc'
% gibt das Tabellenverzeichnis im Inhaltsverzeichnis (toc=Table% of Content) aus.
% Da es aber verschiedene Regelungen je nach Dozent geben kann, werden hier
% beide Varianten dargestellt.
%\documentclass[12pt,oneside,titlepage,listof=totoc,bibliography=totoc]{scrartcl}
\documentclass[12pt,oneside,titlepage,bibliography=totoc]{scrartcl}
\usepackage[utf8]{inputenc}
%Dokumentensprache
\newif\ifde
\newif\ifen

%-----------------------------------
% Meta informationen
%-----------------------------------
%-----------------------------------
% Meta Informationen zur Arbeit
%-----------------------------------

% Autor
\newcommand{\myAutor}{XXX}

% Adresse
\newcommand{\myAdresse}{Stra\ss e 7\\ \> \> \> XXX Mannheim}

% Titel der Arbeit
\newcommand{\myTitel}{Explain the phenomenon of network externalities by means of an example. Use this concept to argue why there is only one Facebook.}

% Betreuer
\newcommand{\myBetreuer}{Prof. Dr. XXXXX}

% Lehrveranstaltung
\newcommand{\myLehrveranstaltung}{Economics}

% Matrikelnummer
\newcommand{\myMatrikelNr}{XXXX}

% Ort
\newcommand{\myOrt}{Mannheim}

% Datum der Abgabe
\newcommand{\myAbgabeDatum}{18.01.2020}

% Semesterzahl
\newcommand{\mySemesterZahl}{1}

% Name der Hochschule
\newcommand{\myHochschulName}{FOM Hochschule für Oekonomie \& Management Essen}

% Standort der Hochschule
\newcommand{\myHochschulStandort}{Mannheim}

% Studiengang
\newcommand{\myStudiengang}{Business Administration (B.A)}

% Art der Arbeit
\newcommand{\myThesisArt}{Assignment Report}

% Zu erlangender akademische Grad
\newcommand{\myAkademischerGrad}{Master of Arts (M.A)}

% Firma
\newcommand{\myFirma}{}


\ifdefined\FOMEN
%Englisch
\detrue
\usepackage[ngerman]{babel}
\else
%Deutsch
\entrue
\usepackage[english]{babel}
\fi




\newcommand{\langde}[1]{%
   \ifde\selectlanguage{ngerman}#1\fi}
\newcommand{\langen}[1]{%
   \ifen\selectlanguage{english}#1\fi}
\langde{\usepackage[babel,german=quotes]{csquotes}}
\langen{\usepackage[babel,english=british]{csquotes}}
\usepackage[T1]{fontenc}
\usepackage{fancyhdr}
\usepackage{fancybox}
\usepackage[a4paper, left=4cm, right=2cm, top=4cm, bottom=2cm]{geometry}
\usepackage{graphicx}
\usepackage{colortbl}
\usepackage[capposition=top]{floatrow}
\usepackage{array}
\usepackage{float}      %Positionierung von Abb. und Tabellen mit [H] erzwingen
\usepackage{footnote}
\usepackage{caption}
\usepackage{enumitem}
\usepackage{amssymb}
\usepackage{mathptmx}
%\usepackage{minted} %Kann für schöneres Syntax Highlighting genutzt werden. ACHTUNG: Python muss installiert sein.
\usepackage[scaled=0.9]{helvet} % Behebt, zusammen mit Package courier, pixelige Überschriften. Ist, zusammen mit mathptx, dem times-Package vorzuziehen. Details: https://latex-kurs.de/fragen/schriftarten/Times_New_Roman.html
\usepackage{courier}
\usepackage{amsmath}
\usepackage[table]{xcolor}
\usepackage{marvosym}			% Verwendung von Symbolen, z.B. perfektes Eurozeichen
\usepackage[colorlinks=true,linkcolor=black]{hyperref}
\definecolor{darkblack}{rgb}{0,0,0}
\hypersetup{colorlinks=true, breaklinks=true, linkcolor=darkblack, menucolor=darkblack, urlcolor=darkblack}
\renewcommand\familydefault{\sfdefault}
\usepackage{ragged2e}

% Mehrere Fussnoten nacheinander mit Komma separiert
\usepackage[hang, multiple]{footmisc}
\setlength{\footnotemargin}{1em}

% todo Aufgaben als Kommentare verfassen für verschiedene Editoren
\usepackage{todonotes}

%Pakete für Tabellen
\usepackage{epstopdf}
\usepackage{nicefrac} % Brüche
\usepackage{multirow}
\usepackage{rotating} % vertikal schreiben
\usepackage{mdwlist}
\usepackage{tabularx}% für breitenangabe

\definecolor{dunkelgrau}{rgb}{0.8,0.8,0.8}
\definecolor{hellgrau}{rgb}{0.0,0.7,0.99}
% Colors for listings
\definecolor{mauve}{rgb}{0.58,0,0.82}
\definecolor{dkgreen}{rgb}{0,0.6,0}

% sauber formatierter Quelltext
\usepackage{listings}
% JavaScript als Sprache definieren:
\lstdefinelanguage{JavaScript}{
	keywords={break, super, case, extends, switch, catch, finally, for, const, function, try, continue, if, typeof, debugger, var, default, in, void, delete, instanceof, while, do, new, with, else, return, yield, enum, let, await},
	keywordstyle=\color{blue}\bfseries,
	ndkeywords={class, export, boolean, throw, implements, import, this, interface, package, private, protected, public, static},
	ndkeywordstyle=\color{darkgray}\bfseries,
	identifierstyle=\color{black},
	sensitive=false,
	comment=[l]{//},
	morecomment=[s]{/*}{*/},
	commentstyle=\color{purple}\ttfamily,
	stringstyle=\color{red}\ttfamily,
	morestring=[b]',
	morestring=[b]"
}

\lstset{
	%language=JavaScript,
	numbers=left,
	numberstyle=\tiny,
	numbersep=5pt,
	breaklines=true,
	showstringspaces=false,
	frame=l ,
	xleftmargin=5pt,
	xrightmargin=5pt,
	basicstyle=\ttfamily\scriptsize,
	stepnumber=1,
	keywordstyle=\color{blue},          % keyword style
  	commentstyle=\color{dkgreen},       % comment style
  	stringstyle=\color{mauve}         % string literal style
}

% Biblatex

%%%% Neuer Leitfaden (2018)
\usepackage[
backend=biber,
style=ext-authoryear,
maxcitenames=2,
maxbibnames=999,
mergedate=false,
date=iso,
seconds=true, %werden nicht verwendet, so werden aber Warnungen unterdrückt.
urldate=iso,
innamebeforetitle,
dashed=false,
autocite=footnote,
doi=false,
mincrossrefs = 1
]{biblatex}%iso dateformat für YYYY-MM-DD

%weitere Anpassungen für BibLaTex
\input{skripte/modsBiblatex2018}

%%%%% Alter Leitfaden. Ggf. Einkommentieren und Bereich hierüber auskommentieren
%\usepackage[
%backend=biber,
%style=numeric,
%citestyle=authoryear,
%url=false,
%isbn=false,
%notetype=footonly,
%hyperref=false,
%sortlocale=de]{biblatex}

%weitere Anpassungen für BibLaTex
%\input{skripte/modsBiblatex}

%%%% Ende Alter Leitfaden

%Bib-Datei einbinden
\addbibresource{literatur/literatur.bib}

%Silbentrennung
\usepackage{hyphsubst}
\HyphSubstIfExists{ngerman-x-latest}{%
\HyphSubstLet{ngerman}{ngerman-x-latest}}{}

% Pfad fuer Abbildungen
\graphicspath{{./}{./abbildungen/}}

%-----------------------------------
% Weitere Ebene einfügen
\input{skripte/weitereEbene}

%-----------------------------------
% Zeilenabstand 1,5-zeilig
%-----------------------------------
\usepackage{setspace}
\onehalfspacing

%-----------------------------------
% Absätze durch eine neue Zeile
%-----------------------------------
\setlength{\parindent}{0mm}
\setlength{\parskip}{0.8em plus 0.5em minus 0.3em}

\sloppy					%Abstände variieren
\pagestyle{headings}

%-----------------------------------
% Abkürzungsverzeichnis
%-----------------------------------
\usepackage[printonlyused]{acronym}

%-----------------------------------
% PDF Meta Daten setzen
%-----------------------------------
\hypersetup{
    pdfinfo={
        Title={\myTitel},
        Subject={\myStudiengang},
        Author={\myAutor},
        Build=1.1
    }
}

%-----------------------------------
% Umlaute in Code korrekt darstellen
% siehe auch: https://en.wikibooks.org/wiki/LaTeX/Source_Code_Listings
%-----------------------------------
\lstset{literate=
	{á}{{\'a}}1 {é}{{\'e}}1 {í}{{\'i}}1 {ó}{{\'o}}1 {ú}{{\'u}}1
	{Á}{{\'A}}1 {É}{{\'E}}1 {Í}{{\'I}}1 {Ó}{{\'O}}1 {Ú}{{\'U}}1
	{à}{{\`a}}1 {è}{{\`e}}1 {ì}{{\`i}}1 {ò}{{\`o}}1 {ù}{{\`u}}1
	{À}{{\`A}}1 {È}{{\'E}}1 {Ì}{{\`I}}1 {Ò}{{\`O}}1 {Ù}{{\`U}}1
	{ä}{{\"a}}1 {ë}{{\"e}}1 {ï}{{\"i}}1 {ö}{{\"o}}1 {ü}{{\"u}}1
	{Ä}{{\"A}}1 {Ë}{{\"E}}1 {Ï}{{\"I}}1 {Ö}{{\"O}}1 {Ü}{{\"U}}1
	{â}{{\^a}}1 {ê}{{\^e}}1 {î}{{\^i}}1 {ô}{{\^o}}1 {û}{{\^u}}1
	{Â}{{\^A}}1 {Ê}{{\^E}}1 {Î}{{\^I}}1 {Ô}{{\^O}}1 {Û}{{\^U}}1
	{œ}{{\oe}}1 {Œ}{{\OE}}1 {æ}{{\ae}}1 {Æ}{{\AE}}1 {ß}{{\ss}}1
	{ű}{{\H{u}}}1 {Ű}{{\H{U}}}1 {ő}{{\H{o}}}1 {Ő}{{\H{O}}}1
	{ç}{{\c c}}1 {Ç}{{\c C}}1 {ø}{{\o}}1 {å}{{\r a}}1 {Å}{{\r A}}1
	{€}{{\EUR}}1 {£}{{\pounds}}1 {„}{{\glqq{}}}1
}

%-----------------------------------
% Kopfbereich / Header definieren
%-----------------------------------
\pagestyle{fancy}
\fancyhf{}
\fancyhead[C]{-\ \thepage\ -}						% Seitenzahl oben, mittg
%\fancyhead[L]{\leftmark}							% kein Footer vorhanden
\renewcommand{\headrulewidth}{0.4pt}


%-----------------------------------
% Start the document here:
%-----------------------------------
\begin{document}

\pagenumbering{Roman}								% Seitennumerierung auf römisch umstellen
\renewcommand{\refname}{\langde{Literaturverzeichnis}
						\langen{Bibliography}}		% "Literatur" in
%"Literaturverzeichnis" umbenennen
\newcolumntype{C}{>{\centering\arraybackslash}X}	% Neuer Tabellen-Spalten-Typ:
%Zentriert und umbrechbar

%-----------------------------------
% Titlepage
%-----------------------------------
\begin{titlepage}
	\newgeometry{left=2cm, right=2cm, top=2cm, bottom=2cm}
	\begin{center}
		\textbf{\myHochschulName}\\
		\textbf{
			\langde{Hochschulzentrum}
			\langen{University Centre}
			\myHochschulStandort}\\
		\vspace{1.5cm}
			\includegraphics[width=3cm]{abbildungen/fomLogo.jpg} \\
		\vspace{1.5cm}
		\langde{Berufsbegleitender Studiengang}
		\langen{Part-time study program}\\
		\myStudiengang, \mySemesterZahl  $ ^{st}  $semester \\
		\vspace{2cm}
		\textbf{\myThesisArt}\\
		\textbf{
				\langde{zur Erlangung des Grades eines}
				\langen{to obtain the degree of}
				}\\
		\textbf{\myAkademischerGrad}\\
		% Oder für Hausarbeiten:
		%\textbf{im Rahmen der Lehrveranstaltung}\\
		%\textbf{\myLehrveranstaltung}\\
		\vspace{2cm}
		\langde{über das Thema}
		\langen{on the subject}\\
		\Large{\myTitel}\\
		\vspace{0.2cm}
	\end{center}
	\normalsize
	\vfill
	\begin{tabbing}
		Links \= Mitte \=Mittez \= Rechts\kill
		\langde{Betreuer}
		\langen{Advisor}: \> \> \>\myBetreuer\\
		\> \> \\

		\langde{Autor}
		\langen{Author}: \> \> \> \myAutor\\
		\> \> \>  \langde{Matrikelnr.}
				\langen{Matriculation}: \myMatrikelNr\\
		\> \> \> \myAdresse\\
		\> \> \>  \\
		\langde{Abgabe}
		\langen{Submission}: \> \> \> \myAbgabeDatum
	\end{tabbing}
\end{titlepage}

%-------Ende Titelseite-------------

%-----------------------------------
% Sperrvermerk
%-----------------------------------
%\input{kapitel/anhang/sperrvermerk}

%-----------------------------------
% Inhaltsverzeichnis
%-----------------------------------
\setcounter{page}{2}
\tableofcontents
\newpage
\setcounter{tocdepth}{2} %wurde vorher in zusaetzlichesMaterial.tex auf 0 gesetzt um Inhalt des Anhangs zu verbergen. Dadurch gehen allerdings Abbildungs und Tabellenverzeichnis kaputt.

%-----------------------------------
% Abbildungsverzeichnis
%-----------------------------------
\listoffigures
\newpage
%-----------------------------------
% Tabellenverzeichnis
%-----------------------------------
\listoftables
\newpage
%-----------------------------------
% Abkürzungsverzeichnis
%-----------------------------------
\input{acronyms}
\newpage
%-----------------------------------
% Seitennummerierung auf arabisch und ab 1 beginnend umstellen
%-----------------------------------
\pagenumbering{arabic}
\setcounter{page}{1}
%-----------------------------------
% Kapitel / Inhalte
%-----------------------------------
\input{kapitel/einleitung/einleitung}
\input{kapitel/kapitel_1/kapitel_1}
\input{kapitel/kapitel_2/kapitel_2}
\input{kapitel/fazit/fazit}

%-----------------------------------
% Literaturverzeichnis
%-----------------------------------
\newpage
%\addcontentsline{toc}{section}{Literatur}

\pagenumbering{Roman} %Zähler wieder römisch ausgeben
\setcounter{page}{4}  %Zähler manuell hochsetzen

\begin{RaggedRight}
\printbibliography
\end{RaggedRight}

% Alternative Darstellung:
% Literaturverzeichnis nach Typ (@book, @arcticle ...) sortiert.
% Dazu die Zeile (\printbibliography) auskommentieren und folgenden code verwenden:

%\printbibheading
%\printbibliography[type=article,heading=subbibliography,title={Artikel}]
%\printbibliography[type=book,heading=subbibliography,title={Bücher}]
%\printbibliography[type=online,heading=subbibliography,title={Webseiten}]

% weitere Variante, die nur Online von anderen Quellen trennt:

%\printbibheading
%\printbibliography[nottype=online,heading=subbibliography,title={Literaturquellen}]
%\printbibliography[type=online,heading=subbibliography,title={Internetquellen}]


% Um mehrere Typen zu vereinen kann ein Bibliography-Filter verwendet werden
% \defbibfilter{literature}{
%   type=article or
%   type=book
% }
% \printbibliography[filter=literature,heading=subbibliography,title={Literatur}]


\fancyhead[C]{\thepage} %weil die Ehrenwörtliche Erklärung keine Seitenzahl hat müssen die Striche entfernt werden
\input{kapitel/anhang/erklaerung}
\end{document}
